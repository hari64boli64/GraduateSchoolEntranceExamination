\documentclass[a4paper, 10pt, dvipdfmx]{jlreq}

\usepackage{amsmath,amsfonts,amssymb}
\usepackage{bm}
\usepackage{mathtools}
\usepackage{siunitx}
\usepackage[dvipdfmx]{graphicx}
\usepackage[dvipdfmx]{color}
\usepackage[dvipdfmx, colorlinks=true, allcolors=blue]{hyperref}
\usepackage{listings, jlisting}
\usepackage{tikz}
\usepackage{physics}
\usepackage{url}

\Urlmuskip=0mu plus 10mu
\allowdisplaybreaks[4]
\frenchspacing
\definecolor{OliveGreen}{rgb}{0.0,0.6,0.0}
\definecolor{Orenge}{rgb}{0.89,0.55,0}
\definecolor{SkyBlue}{rgb}{0.28, 0.28, 0.95}
\lstset{
  language={c++},
  basicstyle={\ttfamily},
  identifierstyle={\small},
  ndkeywordstyle={\small},
  frame=single,
  breaklines=true,
  numbers=left,
  xrightmargin=0zw,
  xleftmargin=3zw,
  numberstyle={\scriptsize},
  lineskip=-0.9ex,
  keywordstyle={\small\bfseries\color{SkyBlue}},  
  commentstyle={\color{OliveGreen}}, 
  stringstyle={\small\ttfamily\color{Orenge}}    
}

\begin{document}

\title{2020年度 大問4}
\author{hari64boli64 (hari64boli64@gmail.com)}
\date{\today}
\maketitle

\section{問題}

\begin{align*}
  \pdv{t}x(n,t) & =x(n-1,t)+x(n+1,t)-2x(n,t) \\
  x(n+N,t)      & =x(n,t)                    \\
  e(m,n)        & =\exp(i\frac{2\pi mn}{N})
\end{align*}

\section{解答}

\subsection*{(1)}

\begin{align*}
                  & \pdv{t}x(n,t)=x(n-1,t)+x(n+1,t)-2x(n,t)                                     \\
  \Leftrightarrow & e(m,n)\dv{t}f_m(t)=e(m,n-1)f_m(t)+e(m,n+1)f_m(t)-2e(m,n)f_m(t)              \\
  \Leftrightarrow & \dv{t}f_m(t)=\qty(\exp(-i\frac{2\pi m}{N})+\exp(i\frac{2\pi m}{N})-2)f_m(t) \\
  \Leftrightarrow & f_m(t)=c_me^{\qty(\exp(-i\frac{2\pi m}{N})+\exp(i\frac{2\pi m}{N})-2)t}     \\
\end{align*}

そして、これは$e(m,n+N)=e(m,n)$より、条件(**)を満たす。

\subsection*{(2)}

(1)の形で書けるとまず仮定して、与えられた初期条件を適用すると、$c_m=\frac{g_n}{e(m,n)}$となる。

しかし、これでは$c_m$が$n$に依存してしまうので、条件に反してしまう。
なので、$c_m$を$n$に依存しないように定めたい。

ここで、$g_n=\sum_{m=0}^{N-1}e(m,n)c'_m$という形で書けることを利用する。

ただし、$c'_m=\frac{1}{N}\sum_{n'=0}^{N-1}e(-m,n')g_{n'}$である。これは離散フーリエ変換に相当する。

なお、このような形で書けることは、以下で確認することが出来る。

\begin{align*}
    & \sum_{m=0}^{N-1}e(m,n)c'_m                                       \\
  = & \frac{1}{N}\sum_{m=0}^{N-1}e(m,n)\sum_{n'=0}^{N-1}e(-m,n')g_{n'} \\
  = & \frac{1}{N}\sum_{n'=0}^{N-1}\sum_{m=0}^{N-1}e(m,n)e(-m,n')g_{n'} \\
  = & \frac{1}{N}\sum_{n'=0}^{N-1}\sum_{m=0}^{N-1}e(m,n-n')g_{n'}      \\
  = & \frac{1}{N}\sum_{n'=0}^{N-1}N\delta_{n,n'}g_{n'}                 \\
  = & \frac{1}{N}Ng_n                                                  \\
  = & g_n                                                              \\
\end{align*}

この事を利用すると、

\begin{align*}
  x(n,t)=\sum_{m=0}^{N-1}e(m,n)f_m(t) \quad (c_m=c'_m)
\end{align*}

とすれば、

\begin{align*}
  x(n,0)=\sum_{m=0}^{N-1}e(m,n)c_m=g_n
\end{align*}

となり、特に、(1)より、これは条件(*),(**)を共に満たす。

よって、

\begin{align*}
  x(n,t) & =\sum_{m=0}^{N-1}e(m,n)f_m(t)                                                                                                        \\
         & =\sum_{m=0}^{N-1}e(m,n)\qty(\frac{1}{N}\sum_{n'=0}^{N-1}e(-m,n')g_{n'})e^{\qty(\exp(-i\frac{2\pi m}{N})+\exp(i\frac{2\pi m}{N})-2)t} \\
\end{align*}

が解となる。

\subsection*{(3)}

\begin{align*}
  \lim_{t\to\infty}x(n,t) & =\lim_{t\to\infty}\sum_{m=0}^{N-1}e(m,n)c_me^{\qty(\exp(-i\frac{2\pi m}{N})+\exp(i\frac{2\pi m}{N})-2)t} \\
                          & =c_0                                                                                                     \\
                          & =\frac{1}{N}\sum_{n=0}^{N-1}e(0,n)g_n                                                                    \\
                          & =\frac{1}{N}\sum_{n=0}^{N-1}g_n                                                                          \\
\end{align*}

\section{知識}

離散フーリエ変換

\begin{thebibliography}{9}
  \bibitem{site:1}
  金沢大学."離散フーリエ変換の計算法".\url{http://leo.ec.t.kanazawa-u.ac.jp/~nakayama/edu/file/signal_proc_ch4.pdf}
  \bibitem{site:2}
  Wikipedia."離散フーリエ変換".2022年12月28日.\url{https://ja.wikipedia.org/wiki/%E9%9B%A2%E6%95%A3%E3%83%95%E3%83%BC%E3%83%AA%E3%82%A8%E5%A4%89%E6%8F%9B}
\end{thebibliography}
\end{document}
