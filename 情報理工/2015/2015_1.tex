\documentclass[a4paper, 10pt, dvipdfmx]{jlreq}

\usepackage{amsmath,amsfonts,amssymb}
\usepackage{bm}
\usepackage{mathtools}
\usepackage{siunitx}
\usepackage[dvipdfmx]{graphicx}
\usepackage[dvipdfmx]{color}
\usepackage[dvipdfmx, colorlinks=true, allcolors=blue]{hyperref}
\usepackage{listings, jlisting}
\usepackage{tikz}
\usepackage{physics}
\usepackage{url}

\Urlmuskip=0mu plus 10mu
\allowdisplaybreaks[4]
\frenchspacing
\definecolor{OliveGreen}{rgb}{0.0,0.6,0.0}
\definecolor{Orenge}{rgb}{0.89,0.55,0}
\definecolor{SkyBlue}{rgb}{0.28, 0.28, 0.95}
\lstset{
  language={c++},
  basicstyle={\ttfamily},
  identifierstyle={\small},
  ndkeywordstyle={\small},
  frame=single,
  breaklines=true,
  numbers=left,
  xrightmargin=0zw,
  xleftmargin=3zw,
  numberstyle={\scriptsize},
  lineskip=-0.9ex,
  keywordstyle={\small\bfseries\color{SkyBlue}},  
  commentstyle={\color{OliveGreen}}, 
  stringstyle={\small\ttfamily\color{Orenge}}    
}

\begin{document}

\title{2015年度 大問1}
\author{hari64boli64 (hari64boli64@gmail.com)}
\date{\today}
\maketitle

\section{問題}

\begin{align*}
  \text{maximize}_{\bm{u},\bm{v}} \quad & \bm{a}^T(\bm{u}-\bm{v})              \\
  \text{subject to} \quad               & \bm{1}^T\bm{u}+\bm{1}^T\bm{v} \leq 1 \\
                                        & \bm{u}\geq \bm{0},\bm{v}\geq \bm{0}  \\
\end{align*}


\begin{align*}
  \text{minimize}_{\bm{x}} \quad & \bm{x}^TQ^{-1}\bm{x}    \\
  \text{subject to} \quad        & ||\bm{x}||_{\infty} = 1 \\
\end{align*}

\section{解答}

\subsection*{(1)}

\subsubsection*{(1-1)}

自明

\subsubsection*{(1-2)}

todo

\subsubsection*{(1-3)}

自明

\subsection*{(2)}

\subsubsection*{(2-1)}

{\color{lightgray}

まず、$||\bm{x}||_2=1$という制約に変更した問題を解く。

$Q=P^TDP$と分解する。ただし、$P$は直交行列で、$D$は対角行列であり、

\begin{align*}
  D=\mqty(d_1 &  & \\ & \ddots & \\ & & d_n)
\end{align*}

とする。

$Q^{-1}=P^TD^{-1}P$である。なお、正定値性より、$D^{-1}$は定義可能である。

その上で、

\begin{align*}
  \bm{x}^TQ^{-1}\bm{x} & = \bm{x}^TP^TD^{-1}P\bm{x}                                                      \\
                       & = (P\bm{x})^TD^{-1}(P\bm{x})                                                    \\
                       & = \bm{y}^TD^{-1}\bm{y} \quad (\bm{y}=P\bm{x},||\bm{y}||_2=\bm{x}^TP^TP\bm{x}=1) \\
                       & = \sum_{i=1}^{n}\frac{y_i^2}{d_i}                                               \\
\end{align*}

となるので、この制約下における解は、$Q$の最大固有値を$d_{\text{max}}$として、$\frac{1}{d_{\text{max}}}$となる。

}

……が、この議論は恐らく関係ない。

ラグランジュの未定乗数法の考え方を用いる。

まず、$x_k=1(1\leq k \leq n)$という制約に変更した問題を解く。

\begin{align*}
  \pdv{L}{\bm{x}} & =\pdv{\bm{x}} \qty(\bm{x}^TQ^{-1}\bm{x}-\lambda(x_k-1)) \\
                  & = 2Q^{-1}\bm{x}-\lambda \bm{e}_k                        \\
                  & = 0
\end{align*}

より、

\begin{align*}
  \bm{x}  & =\frac{\lambda}{2}Q\bm{e}_k \\
  x_k     & =\frac{\lambda}{2}q_{kk}    \\
          & =1                          \\
  \lambda & =\frac{2}{q_{kk}}
\end{align*}

と置くと、

\begin{align*}
  \bm{x}^TQ^{-1}\bm{x} & =\frac{1}{q_{kk}^2}\bm{e}_k^TQ^TQ^{-1}Q\bm{e}_k \\
                       & =\frac{1}{q_{kk}^2}\bm{e}_k^TQ\bm{e}_k          \\
                       & =\frac{1}{q_{kk}^2}q_{kk}                       \\
                       & =\frac{1}{q_{kk}}
\end{align*}

となる。よって、最小値の候補は、$Q$の対角成分の最大値を$d_{\text{max}}=d_k$として、$\frac{1}{d_{\text{max}}}$となる。
特に、これより大きい解は存在しない(ここの議論が本来もう少しきちんとすべき)ので、もしこのような解が存在すれば、これが最適解である。

そして、このような解が、変更された制約下のみならず、元の制約下においても存在することは、

\begin{align*}
  \bm{x} & =\frac{1}{q_{kk}}Q\bm{e}_k   \\
         & =\mqty(\frac{q_{1k}}{q_{kk}} \\ \vdots \\ 1 \\ \vdots \\ \frac{q_{nk}}{q_{kk}})\\
\end{align*}

が、$||\bm{x}||_\infty=1$を満たすことから分かる。

最後の部分で、$\bm{y}=\bm{e}_i+\bm{e}_k,\bm{z}=-\bm{e}_i+\bm{e}_k$に対して、$Q$が正定値対称行列なことを用いると、

\begin{align*}
              & \begin{cases}
    q_{ii}+q_{kk}+2q_{ik} \geq 0 \\
    q_{ii}+q_{kk}-2q_{ik} \geq 0 \\
  \end{cases}                                                         \\
  \Rightarrow & 1 \geq \frac{q_{ii}+q_{kk}}{2} \geq  q_{ik} \geq -\frac{q_{ii}+q_{kk}}{2} \geq -1
\end{align*}

となることを用いた。

\subsubsection*{(2-2)}

\begin{align*}
  C=d_{\text{max}}
\end{align*}

\section{知識}

\subsection{ラグランジュの未定乗数法}

\begin{align*}
  \max_{x,y} \quad  & f(x,y)     \\
  \text{s.t.} \quad & g(x,y) = 0 \\
\end{align*}

この問題は、以下のように解ける。

\begin{align*}
  L(x,y,\lambda) = f(x,y) - \lambda g(x,y)
\end{align*}

とおくと、$(\alpha,\beta)$が極値を与えるならば、$(\alpha,\beta)$は

\begin{align*}
  \pdv{L}{x} = \pdv{L}{y} = \pdv{L}{\lambda} = 0
\end{align*}

の解、または、

\begin{align*}
  \pdv{g}{x} = \pdv{g}{y} = 0
\end{align*}

の解。\cite{site:1}

\begin{thebibliography}{9}
  \bibitem{site:1}
  高校数学の美しい物語.“ラグランジュの未定乗数法と例題”.2021年3月7日.\url{https://manabitimes.jp/math/879}
\end{thebibliography}


\end{document}
